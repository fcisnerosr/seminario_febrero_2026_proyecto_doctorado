\documentclass[aspectratio=169]{beamer}

% Paquetes
\usepackage[utf8]{inputenc}
\usepackage[spanish]{babel}
\usepackage{graphicx}
\usepackage{amsmath}
\usepackage{booktabs}
\usepackage{subcaption}

% Tema
\usetheme{Madrid}
\useoutertheme[subsection=false]{miniframes} % Agrega la barra de navegacion con "bolitas"

% Metadatos
\title[SHM en Plataformas Jacket]{Desarrollo de una metodologia para la deteccion de daño en plataformas marinas fijas por medio de analisis de vibraciones}
\subtitle{Evaluacion de Desempeño - Trabajo de Investigacion de Doctorado}
\author{M. I. Francisco Cisneros}
\institute{
    Coordinacion Academica del Posgrado \\
    Direccion de Desarrollo de Talento \\
    \vspace{0.2cm}
    \textbf{Directores:} Dr. Ivan Felix Gonzalez y Dr. Rolando Salgado Estrada
}
\date{19 de Febrero de 2026}

% Configuracion de imagenes
\graphicspath{{./figs/}}

\begin{document}

% Portada Personalizada
\begin{frame}[plain]
    \centering
    \includegraphics[height=1.3cm]{000_logo.png}
    \vspace{0.2cm}

    \vspace{0.1cm}
    {\footnotesize Coordinacion Academica del Posgrado} \\
    {\footnotesize Direccion de Desarrollo de Talento}

    \vspace{0.6cm}

    {\bfseries \textit{EVALUACION DE DESEMPEÑO}} \\
    \vspace{0.1cm}
    Trabajo de Investigacion de Doctorado \\
    {\footnotesize 8vo Semestre}

    \vspace{0.4cm}

    \begin{beamercolorbox}[sep=8pt,center,shadow=true,rounded=true]{title}
        \usebeamerfont{title}Desarrollo de una metodologia para la deteccion de daño en plataformas marinas fijas por medio de analisis de vibraciones\par
    \end{beamercolorbox}

    \vspace{0.4cm}

    \textbf{M. I. Francisco Cisneros} \\
    {\footnotesize Ingeniero Civil}

    \vfill

    {\footnotesize
    Directores: Dr. Ivan Felix Gonzalez y Dr. Rolando Salgado Estrada \\
    \vspace{0.1cm}
    Periodo: Verano 2026
    }
    \vspace{0.2cm}
\end{frame}

% Indice
\begin{frame}{Contenido}
    \tableofcontents
\end{frame}

% Seccion 2: Justificacion
\section{Justificacion}
\begin{frame}{Justificacion del Problema}
    \begin{columns}
        \column{0.5\textwidth}
        \begin{itemize}
            \item \textbf{Infraestructura Envejecida:} Gran parte de las plataformas en el Golfo de Mexico han superado su vida util de diseño.
            \item \textbf{Costos y Riesgos:} Las inspecciones tradicionales (buzos, ROV) son costosas y peligrosas.
            \item \textbf{Limitaciones Actuales:} Dificultad para detectar daños internos o en etapas tempranas mediante inspeccion visual.
            \item \textbf{Necesidad:} Estrategias de mantenimiento basadas en datos (SHM).
        \end{itemize}
        
        \column{0.5\textwidth}
        \begin{figure}
            \centering
            \includegraphics[width=0.8\linewidth]{001_plataformas_con_abolladuras_al_borde_copaso.png}
            \caption{Deterioro en plataformas marinas.}
        \end{figure}
    \end{columns}
\end{frame}

\begin{frame}{Deterioro Estructural}
    \begin{figure}
        \centering
        \begin{subfigure}[b]{0.45\textwidth}
            \includegraphics[width=\textwidth]{002_elemento_tubular_con_corrosion_excesiva.png}
            \caption{Corrosion severa.}
        \end{subfigure}
        \hfill
        \begin{subfigure}[b]{0.45\textwidth}
            \includegraphics[width=\textwidth]{003_elemento_tubular_con_abolladura_excesiva.png}
            \caption{Abolladura por impacto.}
        \end{subfigure}
    \end{figure}
\end{frame}

% Seccion 3: Objetivos
\section{Objetivos}
\begin{frame}{Objetivos de la Investigacion}
    \begin{itemize}
        \item \textbf{Objetivo General:} Desarrollar y validar una metodologia SHM basada en optimizacion inversa con AG para identificar daño en plataformas Jacket.
        \item \textbf{Objetivos Especificos:}
        \begin{itemize}
            \item Formular el \textbf{Indice de Calidad de Deteccion (ICD)} para fusionar indicadores vibratorios.
            \item Validar la metodologia mediante modelos de elemento finito calibrados.
            \item Evaluar la sensibilidad ante escenarios de corrosión y abolladura.
            \item Determinar el comportamiento de elementos "fusibles" vs "principales".
        \end{itemize}
    \end{itemize}
\end{frame}

% Seccion 4: Marco Teorico
\section{Marco Teorico}
\begin{frame}{Marco Teorico: Elementos de Masa}
    \begin{columns}
        \column{0.6\textwidth}
        La matriz de masa global incluye:
        \begin{itemize}
            \item \textbf{Masa Estructural ($M_s$):} Acero de la plataforma.
            \item \textbf{Masa Adherida ($M_a$):} Volumen de agua que se acelera con la estructura ($C_m = 1.2$ a $1.6$).
            \item \textbf{Masa Atrapada ($M_t$):} Agua contenida en elementos inundados (pilotes/piernas).
        \end{itemize}
        \vspace{0.5cm}
        Impacto del \textbf{Crecimiento Marino}:
        \begin{itemize}
            \item Crecimiento de organismos en la superficie.
            \item Aumenta diametro efectivo y coeficientes hidrodinamicos.
        \end{itemize}
        
        \column{0.4\textwidth}
        \begin{figure}
            \includegraphics[width=0.9\linewidth]{001_modelo_plataforma_con_elemento_tubular_masa_adherida_y_masa_atrapada.png}
        \end{figure}
        \begin{figure}
            \includegraphics[width=0.9\linewidth]{001_modelo_plataforma_con_elemento_tubular_masa_de_crecimiento_marino.png}
        \end{figure}
    \end{columns}
\end{frame}

\begin{frame}{Mecanica del Daño Simulado}
    \begin{columns}
        \column{0.5\textwidth}
        \textbf{Corrosion Uniforme:}
        \begin{itemize}
            \item Reduccion del espesor de pared ($t$).
            \item Impacta area ($A$) e inercia ($I$).
            \item Modulo de elasticidad ($E$) constante.
        \end{itemize}
        \vspace{0.5cm}
        \textbf{Abolladuras:}
        \begin{itemize}
            \item Distorsion de la seccion transversal (aplanamiento).
            \item Reduccion drastica del momento de inercia local ($I_{red}$).
            \item Suavizado numerico con interpolacion polinomica.
        \end{itemize}
        
        \column{0.5\textwidth}
        \begin{figure}
            \includegraphics[width=0.8\linewidth]{004_caracterizacion_de_como_se_reduce_el_espesor_elemento_tubular_por_corrosion.png}
            \caption{Reduccion de espesor por corrosion.}
        \end{figure}
        \begin{figure}
            \includegraphics[width=0.5\linewidth]{004_seccion_transversal_de_elemento_tubular_parametros_abolladura.png}
            \caption{Seccion abollada.}
        \end{figure}
    \end{columns}
\end{frame}

% Seccion 5: Hipotesis
\section{Hipotesis}
\begin{frame}{Hipotesis de Trabajo}
    \begin{quote}
        "Los elementos secundarios (diagonales o \textit{braces}) actuan como fusibles estructurales, manifestando cambios modales detectables a traves del ICD antes de que se comprometa la integridad global de las piernas principales."
    \end{quote}
    \vspace{0.5cm}
    \textbf{Implicacion Operativa:}
    \begin{itemize}
        \item Monitoreo automatico continuo para elementos secundarios.
        \item Inspeccion visual enfocada en nodos criticos y componentes principales.
    \end{itemize}
\end{frame}

% Seccion 6 y 7: Metodologia y Flujo
\section{Metodologia y Flujo de Trabajo}
\begin{frame}{Metodologia Propuesta}
    \begin{figure}
        \centering
        \includegraphics[width=0.75\linewidth]{005_metogologia_propuesta_para_identificar_daños_en_plataformas_reales.jpg}
        \caption{Esquema general de la metodologia SHM.}
    \end{figure}
\end{frame}

\begin{frame}{Optimizacion con Algoritmos Geneticos}
    \begin{figure}
        \centering
        \includegraphics[width=0.7\linewidth]{006_metodología_del_AG_para_localizar_daños_a_nivel_computacional.jpg}
        \caption{Flujo del Algoritmo Genetico.}
    \end{figure}
\end{frame}

\section{Caso de Estudio}
\begin{frame}{Modelo de Plataforma Jacket}
    \begin{columns}
        \column{0.5\textwidth}
        \textbf{Caracteristicas:}
        \begin{itemize}
            \item Estructura tipo Jacket de 4 patas.
            \item Sistema MDOF discretizado.
            \item Inclusion de masa hidrodinamica e interaccion suelo-estructura.
        \end{itemize}
        
        \column{0.5\textwidth}
        \begin{figure}
            \includegraphics[width=0.6\linewidth]{007_modeo_vista_3d_de_plataforma_tipo_jacket_de_caso_de_estudio.png}
            \caption{Vista 3D del modelo FEM.}
        \end{figure}
    \end{columns}
\end{frame}

\begin{frame}{Detalle del Modelo}
    \begin{figure}
        \centering
        \begin{subfigure}[b]{0.45\textwidth}
            \includegraphics[width=\textwidth]{007_modelo_vista_fronta_de_plataforma_tipo_jacket_de_caso_de_estudio.png}
            \caption{Vista Frontal.}
        \end{subfigure}
        \hfill
        \begin{subfigure}[b]{0.45\textwidth}
            \includegraphics[width=\textwidth]{007_modeo_vista_3d_de_plataforma_tipo_jacket_de_caso_de_estudio_pero_con_colores_para_saber_que_secciones_transversales_estan_ubicados_con_una_tabla.png}
            \caption{Identificacion de Secciones.}
        \end{subfigure}
    \end{figure}
\end{frame}

% Seccion 8: Resultados
\section{Resultados}

\begin{frame}{Resultados: Escenario de Abolladura}
    Introduccion al analisis de sensibilidad del ICD ante abolladuras en diferentes elementos.
\end{frame}

\begin{frame}{Abolladura: Vigas vs Braces}
    \begin{figure}
        \centering
        \begin{subfigure}[b]{0.48\textwidth}
            \includegraphics[width=\textwidth]{figs/resultados/abolladura/001_abolladura_vigas_ICD_vs_porcentaje_de_daño.png}
            \caption{Vigas (Beams)}
        \end{subfigure}
        \hfill
        \begin{subfigure}[b]{0.48\textwidth}
            \includegraphics[width=\textwidth]{figs/resultados/abolladura/002_abolladura_brace_ICD_vs_porcentaje_de_daño.png}
            \caption{Diagonales (Braces)}
        \end{subfigure}
        \caption{Sensibilidad del ICD en elementos horizontales vs diagonales.}
    \end{figure}
\end{frame}

\begin{frame}{Abolladura: Comparativa Global}
    \begin{figure}
        \centering
        \includegraphics[width=0.8\linewidth]{figs/resultados/abolladura/005_abolladura_comparativa_global_de_ICD_vs_daño_por_zona_mudline_medio_y_splash_zone.png}
        \caption{Comparativa global de ICD por zonas (Abolladura).}
    \end{figure}
\end{frame}

\begin{frame}{Resultados: Escenario de Corrosion}
    Evaluacion de la metodologia ante perdida de espesor generalizada.
\end{frame}

\begin{frame}{Corrosion: Elementos Inclinados y Resumen}
    \begin{figure}
        \centering
        \begin{subfigure}[b]{0.48\textwidth}
            \includegraphics[width=\textwidth]{figs/resultados/corrosion/003_corrosion_inclined_ICD_vs_porcenaje_de_daño.png}
            \caption{Elementos Inclinados}
        \end{subfigure}
        \hfill
        \begin{subfigure}[b]{0.48\textwidth}
            \includegraphics[width=\textwidth]{figs/resultados/corrosion/004_corrosion_tipo_de_elemeneto_beam_brace_inclined_led_ICD_vs_porcentaje_de_daño.png}
            \caption{Comparativa por Tipo}
        \end{subfigure}
    \end{figure}
\end{frame}

\begin{frame}{Corrosion: Comparativa Global}
    \begin{figure}
        \centering
        \includegraphics[width=0.8\linewidth]{figs/resultados/corrosion/005_corrosion_comparativa_global_de_ICD_vs_daño_por_zona_mudline_medio_y_splash_zone.png}
        \caption{Comparativa global de ICD por zonas (Corrosion).}
    \end{figure}
\end{frame}

% Seccion 9: Discusion
\section{Discusion}
\begin{frame}{Discusion de Resultados}
    \begin{itemize}
        \item \textbf{Comportamiento Diferenciado:}
        \begin{itemize}
            \item Elementos Secundarios (Diagonales): Detectables con ICD a partir del \textbf{30\% de daño}.
            \item Elementos Principales (Piernas): Requieren severidad \textbf{$>$ 50\%} para identificacion fiable.
        \end{itemize}
        \item \textbf{Fusibles Estructurales:} Las diagonales advierten del deterioro antes de fallos criticos globales.
        \item \textbf{Eficacia del ICD:} Penaliza falsos positivos, mejorando la confianza en la deteccion respecto a metodos tradicionales.
    \end{itemize}
\end{frame}

% Seccion 10: Conclusiones
\section{Conclusiones}
\begin{frame}{Conclusiones Generales}
    \begin{enumerate}
        \item La metodologia basada en AG y el indice ICD permite identificar daño estructural en entornos con incertidumbre.
        \item La modelacion fisica (masa adherida, crecimiento marino) es crucial para representar la dinamica real.
        \item Se valida la hipotesis de monitoreo hibrido:
        \begin{itemize}
            \item \textbf{SHM Automatico:} Para vigilancia continua de elementos secundarios.
            \item \textbf{Inspeccion Focalizada:} Para nodos y piernas principales, optimizando recursos.
        \end{itemize}
    \end{enumerate}
\end{frame}

% Seccion 11: Perspectivas
\section{Perspectivas}
\begin{frame}{Perspectivas Futuras}
    \begin{itemize}
        \item Implementacion en tiempo real con datos de sensores in-situ.
        \item Validacion experimental en tanque de olas.
        \item Extension a estructuras eolicas offshore.
        \item Integracion de algoritmos de Machine Learning hibridos para acelerar la convergencia del AG.
        \item Aplicacion de la metodologia a modelos de plataformas reales.
    \end{itemize}
\end{frame}

% Seccion 12: Bibliografia
\section{Bibliografia}
\begin{frame}{Bibliografia}
    \begin{thebibliography}{9}
        \footnotesize
        \bibitem{mubarak2020} Mubarak et al. (2020). Condition Monitoring of Offshore Platforms.
        \bibitem{goldberg1989} Goldberg, D.E. (1989). Genetic Algorithms in Search, Optimization and Machine Learning.
        \bibitem{api2014} API RP 2A-WSD (2014). Recommended Practice for Planning, Designing and Constructing Fixed Offshore Platforms.
    \end{thebibliography}
\end{frame}

\end{document}