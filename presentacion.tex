\documentclass[aspectratio=169]{beamer}

% Paquetes
\usepackage[utf8]{inputenc}
\usepackage[spanish,es-nodecimaldot]{babel}
\usepackage{graphicx}
\usepackage{amsmath}
\usepackage{booktabs}
\usepackage{subcaption}

% Tema
\usetheme{Madrid}
\useoutertheme[subsection=false]{miniframes} % Agrega la barra de navegacion con "bolitas"

% Metadatos
\title[SHM en Plataformas Jacket]{Desarrollo de una metodologia para la deteccion de daño en plataformas marinas fijas por medio de analisis de vibraciones}
\subtitle{Evaluacion de Desempeño - Trabajo de Investigacion de Doctorado}
\author{M. I. Francisco Cisneros}
\institute[]{
    Coordinacion Academica del Posgrado \\
    Direccion de Desarrollo de Talento \\
    \vspace{0.2cm}
    \textbf{Directores:} Dr. Ivan Felix Gonzalez y Dr. Rolando Salgado Estrada
}
\date{19 de Febrero de 2026}

% Configuracion de imagenes
\graphicspath{{./figs/}}

\begin{document}

% Portada Personalizada
\begin{frame}[plain]
    \centering
    \includegraphics[height=1.3cm]{000_logo.png}
    \vspace{0.2cm}

    \vspace{0.1cm}
    {\footnotesize Coordinacion Academica del Posgrado} \\
    {\footnotesize Direccion de Desarrollo de Talento}

    \vspace{0.6cm}

    {\bfseries \textit{EVALUACION DE DESEMPEÑO}} \\
    \vspace{0.1cm}
    Trabajo de Investigacion de Doctorado \\
    {\footnotesize 8vo Semestre}

    \vspace{0.4cm}

    \begin{beamercolorbox}[sep=8pt,center,shadow=true,rounded=true]{title}
        \usebeamerfont{title}Desarrollo de una metodologia para la deteccion de daño en plataformas marinas fijas por medio de analisis de vibraciones\par
    \end{beamercolorbox}

    \vspace{0.4cm}

    \textbf{M. I. Francisco Cisneros} \\
    {\footnotesize Ingeniero Civil}

    \vfill

    {\footnotesize
    Directores: Dr. Ivan Felix Gonzalez y Dr. Rolando Salgado Estrada \\
    \vspace{0.1cm}
    Periodo: Verano 2026
    }
    \vspace{0.2cm}
\end{frame}

\begin{frame}{Tipos de componentes en plataformas}
    \begin{figure}
        \centering
        \includegraphics[width=0.7\linewidth]{figs/000_01_plataforma_marina_subestructura_superestructura.png}
        \caption{Subestructura y Superestructura.}
    \end{figure}
\end{frame}

\begin{frame}{Tipos de plataformas}
    \begin{figure}
        \centering
        \includegraphics[width=0.8\linewidth]{figs/000_02_tipos_de_plataformas.png}
        \caption{Diversidad de plataformas marinas.}
    \end{figure}
\end{frame}

% Seccion 2: Justificacion
\section{Justificacion}
\begin{frame}{Justificacion del Problema}
    \begin{columns}
        \column{0.5\textwidth}
        \begin{itemize}
            \item \textbf{Infraestructura Envejecida:} Gran parte de las plataformas en el Golfo de Mexico han superado su vida util de diseño.
            \item \textbf{Costos y Riesgos:} Las inspecciones tradicionales (buzos, ROV) son costosas y peligrosas.
            \item \textbf{Limitaciones Actuales:} Dificultad para detectar daños internos o en etapas tempranas mediante inspeccion visual.
            \item \textbf{Necesidad:} Estrategias de mantenimiento basadas en datos (SHM).
        \end{itemize}
        
        \column{0.5\textwidth}
        \begin{figure}
            \centering
            \includegraphics[width=0.8\linewidth]{001_plataformas_con_abolladuras_al_borde_copaso.png}
            \caption{Deterioro en plataformas marinas.}
        \end{figure}
    \end{columns}
\end{frame}

\begin{frame}{Colapso y Daño Estructural Severo}
    \begin{figure}
        \centering
        \includegraphics[width=0.75\linewidth]{figs/000_03_danos_en_plataformas_reales.png}
        \caption{Ejemplos de falla global y colapso de plataformas.}
    \end{figure}
\end{frame}

\begin{frame}{Detalle de Corrosion en Elementos Tubulares}
    \begin{figure}
        \centering
        \includegraphics[width=0.75\linewidth]{figs/000_04_dano_real_corrosion_2.png}
        \caption{Corrosion severa y perdida de material en zonas de marea.}
    \end{figure}
\end{frame}

\begin{frame}{Perdida de Espesor y Perforacion}
    \begin{figure}
        \centering
        \includegraphics[width=0.6\linewidth]{figs/000_04_dano_real_corrosion.png}
        \caption{Fractura de elemento diagonal por perdida critica de espesor.}
    \end{figure}
\end{frame}

\begin{frame}{Proceso de Inspeccion Submarina}
    \begin{figure}
        \centering
        \includegraphics[width=0.6\linewidth]{figs/000_06_actualmente_se_realizan_inspecciones_con_buzos.png}
        \caption{Inspeccion visual detallada realizada por buzos.}
    \end{figure}
\end{frame}

\begin{frame}{Interaccion Humana en Entorno Submarino}
    \begin{figure}
        \centering
        \includegraphics[width=0.75\linewidth]{figs/000_06_actualmente_se_realizan_inspecciones_con_buzos2.png}
        \caption{Desafios logisticos y operativos de la inspeccion in-situ.}
    \end{figure}
\end{frame}

% Seccion 3: Objetivos
\section{Objetivos}
\begin{frame}{Objetivos de la Investigacion}
    \begin{itemize}
        \item \textbf{Objetivo General:} Desarrollar y validar una metodologia SHM basada en optimizacion inversa con AG para identificar daño en plataformas Jacket.
        \item \textbf{Objetivos Especificos:}
        \begin{itemize}
            \item Formular el \textbf{Indice de Calidad de Deteccion (ICD)} para fusionar indicadores vibratorios.
            \item Validar la metodologia mediante modelos de elemento finito calibrados.
            \item Evaluar la sensibilidad ante escenarios de corrosión y abolladura.
            \item Determinar el comportamiento de elementos "fusibles" vs "principales".
        \end{itemize}
    \end{itemize}
\end{frame}

% Seccion 4: Marco Teorico
\section{Marco Teorico}
\begin{frame}{Marco Teorico: Elementos de Masa}
    \begin{columns}
        \column{0.6\textwidth}
        La matriz de masa global incluye:
        \begin{itemize}
            \item \textbf{Masa Estructural ($M_s$):} Acero de la plataforma.
            \item \textbf{Masa Adherida ($M_a$):} Volumen de agua que se acelera con la estructura ($C_m = 1.2$ a $1.6$).
            \item \textbf{Masa Atrapada ($M_t$):} Agua contenida en elementos inundados (pilotes/piernas).
        \end{itemize}
        \vspace{0.5cm}
        Impacto del \textbf{Crecimiento Marino}:
        \begin{itemize}
            \item Crecimiento de organismos en la superficie.
            \item Aumenta diametro efectivo y coeficientes hidrodinamicos.
        \end{itemize}
        
        \column{0.4\textwidth}
        \begin{figure}
            \includegraphics[width=0.9\linewidth]{001_modelo_plataforma_con_elemento_tubular_masa_adherida_y_masa_atrapada.png}
        \end{figure}
        \begin{figure}
            \includegraphics[width=0.9\linewidth]{001_modelo_plataforma_con_elemento_tubular_masa_de_crecimiento_marino.png}
        \end{figure}
    \end{columns}
\end{frame}

\begin{frame}{Mecanica del Daño: Corrosion Uniforme}
    \begin{columns}
        \column{0.5\textwidth}
        \begin{itemize}
            \item Reduccion del espesor de pared ($t$).
            \item Impacta area ($A$) e inercia ($I$).
            \item Modulo de elasticidad ($E$) constante.
        \end{itemize}
        
        \column{0.5\textwidth}
        \begin{figure}
            \includegraphics[width=0.8\linewidth]{004_caracterizacion_de_como_se_reduce_el_espesor_elemento_tubular_por_corrosion.png}
            \caption{Reduccion de espesor por corrosion.}
        \end{figure}
    \end{columns}
\end{frame}

\begin{frame}{Mecanica del Dano: Abolladuras}
    \begin{columns}
        \column{0.5\textwidth}
        \begin{itemize}
            \item Distorsion de la seccion transversal (aplanamiento).
            \item Reduccion drastica del momento de inercia local ($I_{red}$).
            \item Suavizado numerico con interpolacion polinomica.
        \end{itemize}
        
        \column{0.5\textwidth}
        \begin{figure}
            \includegraphics[width=0.7\linewidth]{004_seccion_transversal_de_elemento_tubular_parametros_abolladura.png}
            \caption{Seccion abollada.}
        \end{figure}
    \end{columns}
\end{frame}

\begin{frame}{Aportación Novedosa: Formulación Matemática del ICD}
    % Se cambió el título del bloque para resaltar la novedad
    \begin{block}{Contribución Metodológica: \textbf{Indice de Calidad de Detección}}
        Se propone una \textbf{métrica inédita} que integra tres factores normalizados para evaluar la calidad de la solución:
        \begin{equation}
            \text{ICD} = D \times C_{\text{norm}}(\delta) \times P_{\text{FP}}(N_{\text{FP}})
        \end{equation}
        \vspace{0.1cm}
        \centering \small \textbf{Interpretación:} $\text{ICD} \in [0, 1]$ \quad (Donde $1.0 =$ Detección perfecta).
    \end{block}

    \vspace{0.2cm}

    \begin{columns}[t]
        \begin{column}{0.32\textwidth}
            \textbf{1. Éxito ($D$)}
            \begin{itemize} \footnotesize
                \item \textbf{1.0:} Exacta.
                \item \textbf{0.5:} Adyacente.
                \item \textbf{0.0:} Fallo.
            \end{itemize}
        \end{column}

        \begin{column}{0.34\textwidth}
            \textbf{2. Confianza ($C_{\text{norm}}$)}
            \begin{itemize} \scriptsize
                \item Escalamiento logarítmico ($\alpha=0.1$).
            \end{itemize}
            \vspace{0.1cm}
            \centering \small
            $\displaystyle \frac{\ln(1 + \alpha \delta)}{\ln(1 + \alpha \delta_{\max})}$
        \end{column}

        \begin{column}{0.30\textwidth}
            \textbf{3. Penalización ($P_{\text{FP}}$)}
            \begin{itemize} \scriptsize
                \item Decaimiento exp. ($\beta=0.15$).
            \end{itemize}
            \vspace{0.1cm}
            \centering \small
            $\displaystyle e^{- \beta N_{\text{FP}}}$
        \end{column}
    \end{columns}
\end{frame}

% Seccion 5: Hipotesis
\section{Hipotesis}
\begin{frame}{Hipotesis de Trabajo}
    \begin{columns}
        \column{0.6\textwidth}
            \begin{quote}
                "Los elementos secundarios (diagonales o \textit{braces}) actuan como fusibles estructurales, manifestando cambios modales detectables a traves del ICD antes de que se comprometa la integridad global de las piernas principales."
            \end{quote}
            \vspace{0.5cm}
            \textbf{Implicacion Operativa:}
            \begin{itemize}
                \item Monitoreo automatico continuo para elementos secundarios.
                \item Inspeccion visual enfocada en nodos criticos y componentes principales.
            \end{itemize}
        
        \column{0.4\textwidth}
            \begin{figure}
                \centering
                \includegraphics[width=\linewidth, height=0.7\textheight, keepaspectratio]{007_modeo_vista_3d_de_plataforma_tipo_jacket_de_caso_de_estudio.png}
                \caption{Modelo de referencia.}
            \end{figure}
    \end{columns}
\end{frame}

% Seccion 6 y 7: Metodologia y Flujo
\section{Metodologia}
\begin{frame}{Metodologia Propuesta}
    \begin{figure}
        \centering
        \includegraphics[width=0.75\linewidth]{005_metogologia_propuesta_para_identificar_daños_en_plataformas_reales.jpg}
        \caption{Esquema general de la metodologia SHM.}
    \end{figure}
\end{frame}

\begin{frame}{Optimizacion con Algoritmos Geneticos}
    \begin{figure}
        \centering
        \includegraphics[width=0.8\linewidth]{figs/000_05_explicacion_de_un_AG_con_jirafas.png}
        \caption{Analogia evolutiva del AG.}
    \end{figure}
\end{frame}

\begin{frame}{Evolucion del AG}
    \begin{figure}
        \centering
        \includegraphics[width=0.45\linewidth]{figs/000_07_AG_1.png}
        \caption{Evolucion de AG para la busqueda de un minimo global en una superficie multivariable}
    \end{figure}
\end{frame}

\begin{frame}{Evolucion del AG}
    \begin{figure}
        \centering
        \includegraphics[width=0.45\linewidth]{figs/000_07_AG_2.png}
        \caption{Evolucion de AG para la busqueda de un minimo global en una superficie multivariable}
    \end{figure}
\end{frame}

\begin{frame}{Evolucion del AG}
    \begin{figure}
        \centering
        \includegraphics[width=0.45\linewidth]{figs/000_07_AG_3.png}
        \caption{Evolucion de AG para la busqueda de un minimo global en una superficie multivariable}
    \end{figure}
\end{frame}

\section{Caso de Estudio}
\begin{frame}{Modelo de Plataforma Jacket}
    \begin{columns}
        \column{0.4\textwidth}
        \textbf{Caracteristicas:}
        \begin{itemize}
            \item Estructura tipo Jacket de 4 patas.
            \item Sistema MDOF discretizado.
            \item Inclusion de masa hidrodinamica e interaccion suelo-estructura.
        \end{itemize}
        
        \column{0.6\textwidth}
        \begin{figure}
            \centering
            \includegraphics[height=0.8\textheight,keepaspectratio]{007_modeo_vista_3d_de_plataforma_tipo_jacket_de_caso_de_estudio.png}
            \caption{Vista 3D del modelo FEM.}
        \end{figure}
    \end{columns}
\end{frame}

\begin{frame}{Detalle del Modelo}
    \begin{figure}
        \centering
        \begin{subfigure}[b]{0.45\textwidth}
            \centering
            \includegraphics[height=0.75\textheight,width=\textwidth,keepaspectratio]{007_modelo_vista_fronta_de_plataforma_tipo_jacket_de_caso_de_estudio.png}
            \caption{Vista Frontal.}
        \end{subfigure}
        \hfill
        \begin{subfigure}[b]{0.45\textwidth}
            \centering
            \includegraphics[height=0.75\textheight,width=\textwidth,keepaspectratio]{007_modeo_vista_3d_de_plataforma_tipo_jacket_de_caso_de_estudio_pero_con_colores_para_saber_que_secciones_transversales_estan_ubicados_con_una_tabla.png}
            \caption{Identificacion de Secciones.}
        \end{subfigure}
    \end{figure}

\end{frame}

\begin{frame}{Caso de Estudio: Propiedades Geométricas}
    \begin{columns}[c]
        \begin{column}{0.5\textwidth}
            \centering
            \textbf{Identificación de secciones:}
            \vspace{0.5cm}
            
            \begin{table}
                \centering
                \renewcommand{\arraystretch}{1.5}
                \begin{tabular}{c c c}
                    \toprule
                    \textbf{Diámetro} & \textbf{Espesor} & \textbf{Identificador} \\
                    \textbf{Ext. (m)} & \textbf{(mm)} & \textbf{Color} \\
                    \midrule
                    2.18 & 38.1 & \textcolor{blue!80}{\rule{1.5cm}{0.5cm}} \\
                    1.98 & 38.1 & \textcolor{violet!80}{\rule{1.5cm}{0.5cm}} \\
                    1.57 & 25.4 & \textcolor{green!80!black}{\rule{1.5cm}{0.5cm}} \\
                    1.57 & 25.4 & \textcolor{red!90}{\rule{1.5cm}{0.5cm}} \\
                    \bottomrule
                \end{tabular}
            \end{table}
        \end{column}
        
        \begin{column}{0.5\textwidth}
            \begin{figure}
                \centering
                \includegraphics[width=\linewidth, height=0.8\textheight, keepaspectratio]{./figs/016_modelo_colores.png}
                \caption{Discretización por tipo de sección transversal.}
            \end{figure}
        \end{column}
    \end{columns}
\end{frame}

\begin{frame}{Desglose de Ejecuciones del AG}
    \centering
    
    \textbf{Desglose del número total de ejecuciones del Algoritmo Genético (AG):}
    
    \vspace{0.5cm}
    
    \begin{equation*}
        \text{Total de ejecuciones} = 120 \, \text{elementos} \times 18 \, \text{niveles de daño} = 2160
    \end{equation*}

    \vspace{0.8cm}
    
    Donde los niveles de daño corresponden a:
    \vspace{0.2cm}
    
    \begin{gather*}
        \text{Niveles de daño} = \{5\%, 10\%, 15\%, \dots, 90\%\} \\
        \Rightarrow 18 \, \text{niveles en incrementos de } 5\%
    \end{gather*}
\end{frame}

\begin{frame}{Definición de Niveles de Profundidad}
    \begin{figure}
        \centering
        \includegraphics[width=\linewidth, height=0.7\textheight, keepaspectratio]{./figs/019_niveles_mudline.png}
        \caption{Identificación de zonas: Mudline, Sub 1, Sub 2 y Sub 3.}
    \end{figure}
\end{frame}

\begin{frame}{Recordatorio: Formulación Matemática del ICD}
    \textit{(Recordatorio de la métrica utilizada)}
    
    % Se cambió el título del bloque para resaltar la novedad
    % \begin{block}{Contribución Metodológica: \textbf{Indice de Calidad de Detección}}
        Se propone una \textbf{métrica inédita} que integra tres factores normalizados para evaluar la calidad de la solución:
        \begin{equation}
            \text{ICD} = D \times C_{\text{norm}}(\delta) \times P_{\text{FP}}(N_{\text{FP}})
        \end{equation}
        \vspace{0.1cm}
        \centering \small \textbf{Interpretación:} $\text{ICD} \in [0, 1]$ \quad (Donde $1.0 =$ Detección perfecta).
    \end{block}

    \vspace{0.2cm}

    \begin{columns}[t]
        \begin{column}{0.32\textwidth}
            \textbf{1. Éxito ($D$)}
            \begin{itemize} \footnotesize
                \item \textbf{1.0:} Exacta.
                \item \textbf{0.5:} Adyacente.
                \item \textbf{0.0:} Fallo.
            \end{itemize}
        \end{column}

        \begin{column}{0.34\textwidth}
            \textbf{2. Confianza ($C_{\text{norm}}$)}
            \begin{itemize} \scriptsize
                \item Escalamiento logarítmico ($\alpha=0.1$).
            \end{itemize}
            \vspace{0.1cm}
            \centering \small
            $\displaystyle \frac{\ln(1 + \alpha \delta)}{\ln(1 + \alpha \delta_{\max})}$
        \end{column}

        \begin{column}{0.30\textwidth}
            \textbf{3. Penalización ($P_{\text{FP}}$)}
            \begin{itemize} \scriptsize
                \item Decaimiento exp. ($\beta=0.15$).
            \end{itemize}
            \vspace{0.1cm}
            \centering \small
            $\displaystyle e^{- \beta N_{\text{FP}}}$
        \end{column}
    \end{columns}
\end{frame}


% Seccion 8: Resultados
\section{Resultados}

\begin{frame}{Resultados: Escenario de Abolladura}
    Introduccion al analisis de sensibilidad del ICD ante abolladuras en diferentes elementos.
\end{frame}

\begin{frame}{Abolladura: Vigas vs Braces}
    \begin{figure}
        \centering
        \begin{subfigure}[b]{0.48\textwidth}
            \includegraphics[width=\textwidth]{figs/resultados/abolladura/001_abolladura_vigas_ICD_vs_porcentaje_de_daño.png}
            \caption{Vigas (Beams)}
        \end{subfigure}
        \hfill
        \begin{subfigure}[b]{0.48\textwidth}
            \includegraphics[width=\textwidth]{figs/resultados/abolladura/002_abolladura_brace_ICD_vs_porcentaje_de_daño.png}
            \caption{Diagonales (Braces)}
        \end{subfigure}
        \caption{Sensibilidad del ICD en elementos horizontales vs diagonales.}
    \end{figure}
\end{frame}

\begin{frame}{Abolladura: Comparativa Global}
    \begin{figure}
        \centering
        \includegraphics[width=0.8\linewidth]{figs/resultados/abolladura/005_abolladura_comparativa_global_de_ICD_vs_daño_por_zona_mudline_medio_y_splash_zone.png}
        \caption{Comparativa global de ICD por zonas (Abolladura).}
    \end{figure}
\end{frame}

\begin{frame}{Resultados: Escenario de Corrosion}
    Evaluacion de la metodologia ante perdida de espesor generalizada.
\end{frame}

\begin{frame}{Corrosion: Elementos Inclinados y Resumen}
    \begin{figure}
        \centering
        \begin{subfigure}[b]{0.48\textwidth}
            \includegraphics[width=\textwidth]{figs/resultados/corrosion/003_corrosion_inclined_ICD_vs_porcenaje_de_daño.png}
            \caption{Elementos Inclinados}
        \end{subfigure}
        \hfill
        \begin{subfigure}[b]{0.48\textwidth}
            \includegraphics[width=\textwidth]{figs/resultados/corrosion/004_corrosion_tipo_de_elemeneto_beam_brace_inclined_led_ICD_vs_porcentaje_de_daño.png}
            \caption{Comparativa por Tipo}
        \end{subfigure}
    \end{figure}
\end{frame}

\begin{frame}{Corrosion: Comparativa Global}
    \begin{figure}
        \centering
        \includegraphics[width=0.8\linewidth]{figs/resultados/corrosion/005_corrosion_comparativa_global_de_ICD_vs_daño_por_zona_mudline_medio_y_splash_zone.png}
        \caption{Comparativa global de ICD por zonas (Corrosion).}
    \end{figure}
\end{frame}

% Seccion 9: Discusion
\section{Discusion}
\begin{frame}{Discusion de Resultados}
    \begin{itemize}
        \item \textbf{Comportamiento Diferenciado:}
        \begin{itemize}
            \item Elementos Secundarios (Diagonales): Detectables con ICD a partir del \textbf{30\% de daño}.
            \item Elementos Principales (Piernas): Requieren severidad \textbf{$>$ 50\%} para identificacion fiable.
        \end{itemize}
        \item \textbf{Fusibles Estructurales:} Las diagonales advierten del deterioro antes de fallos criticos globales.
        \item \textbf{Eficacia del ICD:} Penaliza falsos positivos, mejorando la confianza en la deteccion respecto a metodos tradicionales.
        \item \textbf{Alcance del estudio:} La validacion se realizo mediante modelos numericos de alta fidelidad. La experimentacion in-situ no fue factible en esta etapa por limitaciones logisticas y de recursos.
    \end{itemize}
\end{frame}

% Seccion 10: Conclusiones
\section{Conclusiones}
\begin{frame}{Conclusiones Generales}
    \begin{enumerate}
        \item La metodologia basada en AG y el indice ICD permite identificar daño estructural en entornos con incertidumbre.
        \item La modelacion fisica (masa adherida, crecimiento marino) es crucial para representar la dinamica real.
        \item Se valida la hipotesis de monitoreo hibrido:
        \begin{itemize}
            \item \textbf{SHM Automatico:} Para vigilancia continua de elementos secundarios.
            \item \textbf{Inspeccion Focalizada:} Para nodos y piernas principales, optimizando recursos.
        \end{itemize}
    \end{enumerate}
\end{frame}

% Seccion 11: Perspectivas
\section{Perspectivas}
\begin{frame}{Perspectivas Futuras}
    \begin{itemize}
        \item Implementacion en tiempo real con datos de sensores in-situ.
        \item Validacion experimental en tanque de olas.
        \item Extension a estructuras eolicas offshore.
        \item Integracion de algoritmos de Machine Learning hibridos para acelerar la convergencia del AG.
        \item Aplicacion de la metodologia a modelos de plataformas reales.
        \item Validacion en campo: Aunque la metodologia es aplicable, las pruebas en una plataforma real estan pendientes debido a restricciones presupuestarias y de logistica de acceso.
    \end{itemize}
\end{frame}

% Seccion 12: Bibliografia
\section{Bibliografia}
\begin{frame}[allowframebreaks]{Referencias}
    \begin{thebibliography}{99}
        \footnotesize % Reduce un poco el tamano de la fuente para que quepan mas

        \bibitem{aghaeidoost2023}
        Aghaeidoost, V., y Seyedpoor, S. M. (2023). Damage detection in jacket-type offshore platforms via generalized flexibility matrix and optimal genetic algorithm. \textit{Ocean Engineering}, 270, 113568.

        \bibitem{api2014}
        American Petroleum Institute. (2014). Planning, designing, and constructing fixed offshore platforms - working stress design [Manual de software informatico]. Washington, Estados Unidos. Descargado de \url{https://www.api.org/products-and-services/standards/purchase} (IHS bajo licencia de API. Recuperado el 1 de diciembre de 2023).

        \bibitem{cardenas2019}
        Cardenas-Arias, C. G. (2019). Elasticity modulus variation of the AISI SAE 1045 steel subjected to corrosion process by chloride using tension test destructive. \textit{IOP Conference Series: Materials Science and Engineering}, 844(1), 012059. doi: 10.1088/1757-899X/844/1/012059.

        \bibitem{doebling1996}
        Doebling, S. W. (1996). \textit{Damage identification and health monitoring of structural and mechanical systems from changes in their vibration characteristics: A literature review} (Inf. Tec.). Los Alamos, Nuevo Mexico, Estados Unidos: Los Alamos National Laboratory. Descargado 2022-12-16, de \url{https://digital.library.unt.edu/ark:/67531/metadc663932/m1/25/}.

        \bibitem{ghsoub2018}
        Ghsoub, M. B. (2018). \textit{Structural health monitoring of offshore jacket platforms} (Tesis Doctoral, Politecnico di Torino, Torino, Italia). Descargado 2023-07-11, de \url{https://webthesis.biblio.polito.it/view/creators/Ghsoub=3AMarie_Belle=3A=3A.html}.

        \bibitem{goldberg1989}
        Goldberg, D. E. (1989). \textit{Genetic algorithms in search, optimization, and machine learning}. Reading, MA: Addison-Wesley.

        \bibitem{techdiving2020}
        Instituto de buceo comercial tech diving. (2020). \textit{Tech diving}. Descargado 2023-07-11, de \url{http://institutodebuceocomercial.lat/}.

        \bibitem{malekzadeh2013}
        Malekzadeh, H. (2013). Damage detection in an offshore jacket platform using genetic algorithm-based finite element model updating with noisy modal. \textit{Procedia Engineering}, 54, 480-490. doi: 10.1016/j.proeng.2013.03.044.

        \bibitem{mubarak2020}
        Mohamed Mubarak Abdul Wahab, V. J. (2020, 24 de September). Condition assessment techniques for aged fixed-type offshore platforms considering decommissioning: a historical review. \textit{Journal of Marine Science and Application}, 2. doi: 10.1007/s11804-020-00181-z.

        \bibitem{sinergia2024}
        Sinergia. (2024). A pesar de record en accidentes, pemex recorta 50\% el gasto en mantenimiento. \textit{Sinergia Energetica}. Descargado de \url{https://secmexico.com/a-pesar-de-record-en-accidentes-pemex-recorta-50-el-gasto-en-mantenimiento/}.

    \end{thebibliography}
\end{frame}

\end{document}